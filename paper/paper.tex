\documentclass[]{article}
\usepackage{lmodern}
\usepackage{authblk}
\usepackage{amssymb,amsmath}
\usepackage{ifxetex,ifluatex}
\usepackage{fixltx2e} % provides \textsubscript
\ifnum 0\ifxetex 1\fi\ifluatex 1\fi=0 % if pdftex
  \usepackage[T1]{fontenc}
  \usepackage[utf8]{inputenc}
\else % if luatex or xelatex
  \ifxetex
    \usepackage{mathspec}
  \else
    \usepackage{fontspec}
  \fi
  \defaultfontfeatures{Ligatures=TeX,Scale=MatchLowercase}
\fi
% use upquote if available, for straight quotes in verbatim environments
\IfFileExists{upquote.sty}{\usepackage{upquote}}{}
% use microtype if available
\IfFileExists{microtype.sty}{%
\usepackage{microtype}
\UseMicrotypeSet[protrusion]{basicmath} % disable protrusion for tt fonts
}{}
\usepackage[margin=1in]{geometry}
\usepackage{hyperref}
\hypersetup{unicode=true,
            pdftitle={corner.py: Scatterplot matrices in Python},
            pdfborder={0 0 0},
            breaklinks=true}
\urlstyle{same}  % don't use monospace font for urls
\usepackage{graphicx,grffile}
\makeatletter
\def\maxwidth{\ifdim\Gin@nat@width>\linewidth\linewidth\else\Gin@nat@width\fi}
\def\maxheight{\ifdim\Gin@nat@height>\textheight\textheight\else\Gin@nat@height\fi}
\makeatother
% Scale images if necessary, so that they will not overflow the page
% margins by default, and it is still possible to overwrite the defaults
% using explicit options in \includegraphics[width, height, ...]{}
\setkeys{Gin}{width=\maxwidth,height=\maxheight,keepaspectratio}
\IfFileExists{parskip.sty}{%
\usepackage{parskip}
}{% else
\setlength{\parindent}{0pt}
\setlength{\parskip}{6pt plus 2pt minus 1pt}
}
\setlength{\emergencystretch}{3em}  % prevent overfull lines
\providecommand{\tightlist}{%
  \setlength{\itemsep}{0pt}\setlength{\parskip}{0pt}}
\setcounter{secnumdepth}{0}
% Redefines (sub)paragraphs to behave more like sections
\ifx\paragraph\undefined\else
\let\oldparagraph\paragraph
\renewcommand{\paragraph}[1]{\oldparagraph{#1}\mbox{}}
\fi
\ifx\subparagraph\undefined\else
\let\oldsubparagraph\subparagraph
\renewcommand{\subparagraph}[1]{\oldsubparagraph{#1}\mbox{}}
\fi

\title{corner.py: Scatterplot matrices in Python}

    \author{Daniel Foreman-Mackey}
    \affil{Sagan Fellow, University of Washington}
    \date{26 May 2016}

\begin{document}
\maketitle

\textbf{Paper DOI:} \url{http://dx.doi.org/10.21105/joss.00024}\\
\textbf{Software Repository:} \url{https://github.com/dfm/corner.py}\\
\textbf{Software Archive:} \url{http://dx.doi.org/10.5281/zenodo.53155}\\

\section{Summary}\label{summary}

This Python module uses matplotlib (Hunter 2007) to visualize
multidimensional samples using a scatterplot matrix. In these
visualizations, each one- and two-dimensional projection of the sample
is plotted to reveal covariances. \emph{corner} was originally conceived
to display the results of Markov Chain Monte Carlo simulations and the
defaults are chosen with this application in mind but it can be used for
displaying many qualitatively different samples.

Development of \emph{corner} happens on GitHub and any issues can be
raised there (Foreman-Mackey 2016). \emph{corner} has been used
extensively in the astronomical literature and it has occasionally been
cited as \texttt{corner.py} or using its previous name
\texttt{triangle.py}. The source code for \emph{corner} has been
archived to Zenodo and it has the DOI (Zenodo Archive 2016)

The following is a simple demonstration of a visualization made with
\emph{corner}:

-\includegraphics{corner.png}

\section*{References}\label{references}
\addcontentsline{toc}{section}{References}

\hypertarget{refs}{}
\hypertarget{ref-corner_github}{}
Foreman-Mackey, Daniel. 2016. ``Corner.py on Github.''
\url{https://github.com/dfm/corner.py}.

\hypertarget{ref-matplotlib}{}
Hunter, John D. 2007. ``Matplotlib: A 2D Graphics Environment.''
\emph{Computing in Science and Engineering} 9 (3): 90--95.
doi:\href{https://doi.org/10.1109/MCSE.2007.55}{10.1109/MCSE.2007.55}.

\hypertarget{ref-corner_archive}{}
Zenodo Archive. 2016. ``Corner.py: Scatterplot Matrices in Python.''
\url{http://dx.doi.org/10.5281/zenodo.53155}.
doi:\href{https://doi.org/10.5281/zenodo.53155}{10.5281/zenodo.53155}.

\end{document}
